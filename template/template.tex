\documentclass[12pt, a4paper, oneside, UTF8]{ctexart}
\usepackage{amsmath, amsthm, amssymb, bm, graphicx, hyperref, mathrsfs}
\usepackage[backend=bibtex, style=alphabetic]{biblatex}
\usepackage[
	n, % or lambda
	advantage,
	operators,
	sets,
	adversary,
	landau,
	probability,
	notions,
	logic,
	ff,
	mm,
	primitives,
	events,
	complexity,
	oracles,
	asymptotics,
	keys
]{cryptocode}
\usepackage{enumerate}
\usepackage{listings}
\usepackage{xcolor}

% 添加 .bib 文件的命令应该放在这里
\addbibresource{abbrev3.bib}
\addbibresource{crypto.bib}
\addbibresource{biblio.bib}

\title{{\Huge{\textbf{标题}}}\\——副标题}
\author{Ji Li}
\date{\today}
\linespread{1.5}

\begin{document}

\maketitle

\section{小节标题}

这是笔记的正文部分。
\subsection{小节标题}

这是笔记的正文部分。
\section{小节标题}

这是笔记的正文部分。
\section{小节标题}

这是笔记的正文部分。
\section{小节标题}

这是笔记的正文部分。
\section{小节标题}

这是笔记的正文部分。
\section{小节标题}

这是笔记的正文部分。
\section{小节标题}

这是笔记的正文部分。
\section{小节标题}

这是笔记的正文部分。

\section{小节标题}

这是笔记的正文部分。
\section{小节标题}

这是笔记的正文部分。
\section{小节标题}

这是笔记的正文部分。
\section{小节标题}

这是笔记的正文部分。
\section{小节标题}

这是笔记的正文部分。
\section{小节标题}

这是笔记的正文部分。
\section{小节标题}

这是笔记的正文部分。
\section{小节标题}

这是笔记的正文部分。


% 输出参考文献
\printbibliography

\end{document}
